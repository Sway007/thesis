% !Mode:: "TeX:UTF-8" 
\begin{conclusions}
 
\subsection{本文总结}

随着科学技术的发展,通过不断地探索与研究,人类的交通出行方式发生了巨大的变化,出行效率得到里极大的改善,以前只能出现在古人想象中的日行千里在如今人类生活中已变成现实。这其中起到关键作用的是人类交通工具的不断发展与改进。以陆地的交通工具为例,从原始社会的徒步、马演变到如今的火车、各种机动汽车。在当今社会,人类日常生活中最重要,也是最不可缺少的交通工具为汽车,在人工智能技术,特别是深度神经网络技术发展尚未成熟的年代,汽车的控制权都是完全交由人类掌控。由于人类不具有人工智能对四周环境数据的处理能力和应变能力,在各种场景下都不可避免地出现由于人为不规范的驾驶行为所引起的车祸。尤其是在如今经济发展迅猛的年代,越来越多的汽车被制造,出现在公路上,而事故发生的主要原因:人类不规范的驾驶行为在短时间内很难有质的提升,因此每年不断被生产制造的汽车不可避免的增加了车祸发生的数量。汽车交通的安全问题困扰了人类多年,如今得益于深度学习技术的发展,人类研发了自动驾驶技术,将汽车的控制权交给了对环境应变能力和处理能力更为优秀的计算机。但目前自动驾驶技术的发展尚未成熟,在安全性能上仍有较大的提升空间,特别是专门针对自动驾驶系统的测试技术还没有一个很好的解决方案。

本文探讨了已有的自动驾驶系统测试框架DeepTest存在的主要缺陷,在DeepTest和DeepRoad框架的基础上提出了自动驾驶系统蜕变测试框架,并将该框架对现实中的自动驾驶系统进行了系统测试和实验,成功地发现了自动驾驶系统中存在的不一致驾驶行为,揭示了即便在某种场景下,比如晴天,表现良好的自动驾驶系统,在其他天气场景下仍存在着不可忽视的安全隐患。

除此外,我们为了增强DeepRoad框架中的测试用例生成模块,即改善DeepRoad的图像合成功能,提升系统测试用例质量,从而增加测试系统的性能,我们对目前能够实现路况图像在不同天气场景下转换技术的实证研究,主要调研了对抗生成网络和神经风格迁移技术,成功地对8个模型进行了路况图像转换实验,并对每个模型的最终实验数据进行了统计和分析。为了综合评价每个模型在DeepRoad框架中对于测试用例合成性能的优劣,我们总结了3个评价指标,并对每项指标对应8种模型都进行了实验和数据统计,在对最终的统计数据进行观察后分别做了定量和定性分析,总结了以下6点发现:
\begin{enumerate}
    \item 对抗生成网络技术的模型训练时间成本高于神经风格迁移技术
    \item 神经风格迁移技术合成图与原图更接近
    \item 神经风格迁移技术合成图对自动驾驶系统输出干扰更小
    \item 内容图像数据集和风格图像数据集在语义上的相关性对合成图的质量有很大影响
    \item 对抗生成网络模型在图像风格转换的应用上,很容易将风格图像的内容特征重构到最终的合成图中
    \item 变分自编码技术对语义复杂的图像特征的提取和学习能力有限
\end{enumerate}

\subsection{未来展望}

为了降低研究的难度,本文研究的自动驾驶系统是现实中典型自动驾驶系统的简化版本,主要体现在研究自动驾驶系统的控制单元上,即同深度神经网络实现的自动驾驶算法。现实的控制单元考虑的输入除了本文提到的有摄像头拍摄的路况图像外,还有车辆上雷达和红外探测装置接受的数据,神经网络的输出除了方向盘控制信号外还有油门信号和刹车信号等等,因此我们希望未来能完善DeepRoad测试框架,使其对真实的自动驾驶系统也有很强的测试性能。

在增强DeepRoad测试用例合成模块性能的研究中,我们对8中模型使用到的数据集,其数据覆盖范围仍有不足,尚未覆盖到现实中的各种天气情况和驾驶场景,我们希望未来的工作能够扩充已有的数据集,增加测试用例合成模块能够合成的路况场景种类。

此外,对于提高自动驾驶系统测试用例,即路况合成图质量的研究,除了考虑使用深度学习技术自动产生新的测试用例外,还可以在原始数据收集的源头上作改进。比如结合当下手机厂商研发的各种摄像头技术,直接提高车载摄像头拍摄的路况图像质量,使其减少因摄像头、天气、焦距等因素引起的图像失真问题。

我们相信除了本文提到的8中成功实验的深度学习技术,一定还有其他的技术能够实现路况图像合成功能,在未来的工作中我们会继续对目前的深度学习技术进行调研,扩充DeepRoad测试用例合成模块的可替代技术,并将可实现测试用例合成的技术全部合成到DeepRoad框架中,使得DeepRoad能够对各种测试场景自动选择最优的测试用例合成技术。

\end{conclusions}
