% !Mode:: "TeX:UTF-8"

\chapter{背景}[Background]

虽然目前很多公司都开发了自己的自动驾驶系统,但其使用的技术几乎都是基于计算机视觉,目标检测,目标识别等深度学习、机器学习技术。因此,针对自动驾驶系统的测试技术也是源于深度学习系统的测试技术。本章主要讲述传统的深度学习测试技术以及目前学术界比较推崇的自动测试技术。

\section{传统的深度学习系统测试技术}[The traditional testing technology of DNN]

深度学习技术是一种通过研究同类大量数据的表征,对未知新数据的特征进行推测的一门技术。在其行使职能,即预测新数据特征前,必须要学习大量同类的数据,即模型训练。模型训练完成后为了提前检测模型的准确性,会在之前的训练数据集中保留一部分数据,作为训练结束后的模型的测试数据集,使其在未被学习过的测试数据集上进行预测,最后以测试数据集上的准确性作为训练好的模型的精准度。目前学术界公认理想的训练数据集与测试数据集占比分别为70\%和30\%。

% TODO 可扩展

数据集的具体数量跟模型处理的具体问题相关,一般来说,处理的问题越复杂,即数据的特征越多,需要的数据量也就越多,比如比较出名的ImageNet\cite{ImageNet}比赛,公开可用的数据集多达1500万张由人工标注的图片数据。深度学习技术对已有数据特征拟合的本质和其训练测试的过程导致其对数据量的严重依赖,传统的深度学习测试需要大量的人工收集、标注数据,着极度的增加了其中的人力成本。除此之外,传统的通过人工收集数据的方式有严重的缺点,即收集到的数据无法保证覆盖到了所有可能的极端场景,以自动驾驶测试数据集为例,人工收集的数据集一般是车载记录仪记录的道路驾驶视频图片,但一般大雨、大雪等极端天气场景数据很少也很难收集,这就给相应的极端场景自动驾驶系统测试带来了不确定性。

针对上诉问题,DeepXplore和DeepTest提出了深度学习系统测试用例自动生成系统来缓解深度学习系统对于数据量的依赖。

\section{DeepXplore和DeepTest}[DeepXplore and DeepTest]